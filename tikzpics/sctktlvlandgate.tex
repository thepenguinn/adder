\ctikzsubcircuitdef{sctktlvlandgate} {A, B, O, V, G} {
	coordinate(#1-A) -- ++(0.5,0) to[R, l=~, name=#1-r1]
	++(2,0) node(#1-nb1)[npn, anchor=B]{}
	(#1-nb1.E) node(#1-nb2)[npn, anchor=C]{} (#1-nb2.B)
	to[R, l=~, name=#1-r2] ++(-2,0) -- ++(-0.5,0) coordinate(#1-B)
	(#1-nb2.E) to[short, *-] ++(0,-1) coordinate(#1-G)
	(#1-nb1.C) to[short, *-] ++(1,0) node(#1-nb3)[npn, anchor=B]{}
	(#1-nb3.C) to[R, l=~, name=#1-r3] ++(0,2)
	(#1-nb1.C) -- (#1-nb1.C|-#1-nb3.C) to[R, l=~, name=#1-r4] ++(0,2)
	coordinate(#1-Vbr) to[short, *-] (#1-Vbr-|#1-nb3)
	(#1-Vbr) -- ++(0,1) coordinate(#1-V)
	(#1-nb3.C) to[short, *-] ++(2,0) -- ++(0.5,0) coordinate(#1-O)
	(#1-nb2.E) -| (#1-nb3.E)
}

\ctikzsubcircuitactivate{sctktlvlandgate}

% name, baseresistor, outresistor, gatetransistor
\newcommand\sctklabeltlvlandgate[4] {
    (#1-r1.n) node[above]{#2} % baseresistors
	(#1-r2.n) node[below]{#2}
	(#1-r4.n) node[left]{#2}

    (#1-r3.n) node[left]{#3} % outresistor

	(#1-nb1) node[right]{#4} % gatetransistor
	(#1-nb2) node[right]{#4}
	(#1-nb3) node[right]{#4}
}
